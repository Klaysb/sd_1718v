\chapter{Requisitos}
Neste capitulo é descrito os requisitos funcionais do sistema, assim como os não funcionais.
\section{Funcionais}
Os requisitos funcionais deste sistema são os seguintes apresentados:
\begin{itemize}
	\item Evitar sobrecarga no servidor central;
	\item O cliente comunica apenas com o servidor da sua região;
	\item O cliente tem um identificador único;
	\item O cliente conhece todos os servidores regionais;
	\item O cliente regista-se num servidor regional;
	\item O cliente pode enviar mensagens para um único cliente ou para um grupo onde este pertença;
	\item O cliente pode mudar de região;
	\item O cliente pode criar grupos;
	\item O grupo pode ter clientes pertencentes a várias regiões;
	\item Os servidores regionais e centrais devem conhecer a estrutura dos grupos;
	\item Os clientes que não estejam conectados não recebem as mensagens.
\end{itemize}
\section{Não Funcionais}
Os requisitos não funcionais deste sistema são os seguintes apresentados:
\begin{itemize}
	\item A aplicação cliente realizada em WinForm;
	\item Ter um sistema escalável;
	\item Transparência à concorrência no acesso dos servidores;
	\item Suportar vários canais de comunicação (e.g.: HTTP, TCP);
	\item Tratar as falhas de forma a que o sistema continue funcional, mesmo com a existência de erros em um dos componentes.
\end{itemize}