\chapter{Tolerância a Falhas}

Com esta arquitetura existem cenários em que o servidor central possa falhar e as trocas de mensagens entre regiões não aconteça, mas que dentro de cada região continue a funcionar. Se o servidor central voltar a conectar-se, irá realizar uma nova conexão com os servidores regionais que conhece. Esta solução permite a troca de mensagens entre utilizadores dentro da mesma região enquanto não estiver conectado um servidor central.\\

Um outro cenário seria um servidor regional a falhar. Neste caso só a região desse servidor é que não conseguiria trocar mensagens. Mensagens de outras regiões também não chegariam à região afetada. Caso o servidor central envie mensagens para este servidor regional, tal não é possível, pois este servidor foi desconectado. Esta solução tem um inconveniente. O servidor central tem guardado informação de servidores regionais que podem já não estar conectados. O servidor central não tem maneira de identificar se o servidor regional foi desconectado ou se é um problema de comunicação. Se um servidor regional falhar e ficar ativo novamente, irá conseguir conectar-se com o servidor central. O problema desta solução é perder a informação dos utilizadores e dos grupos existentes.\\

Se um cliente desconectar-se sem informar o servidor regional, os dados ficam guardados nesse servidor, pois não há forma de saber que o cliente tem problemas de comunicação ou se está mesmo desconectado. 
Uma solução alternativa passaria por contar o número de chamadas consecutivas que o servidor regional faz ao cliente e este não responde com sucesso. Teria de ser atribuído um número máximo de chamadas consecutivas falhadas, e eliminar a informação deste cliente no servidor regional após o número de chamadas ultrapassar o limite. O problema com esta solução é que o cliente poderia estar novamente disponível após o limite de tentativas, e assim o servidor regional iria apagar informação de um cliente ativo.